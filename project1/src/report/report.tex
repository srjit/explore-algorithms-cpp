\documentclass{article}
%\usepackage[utf8]{inputenc}
\usepackage[a4paper, total={7in, 8in}]{geometry}
\usepackage{spverbatim}
\usepackage{minted}
\usepackage{amssymb}
\usepackage{bm}
\usepackage{graphicx}
\usepackage{listings}
\usepackage{amsmath}
\usepackage{halloweenmath}
\usepackage{enumitem}
\usepackage{setspace}
\usepackage{tabto}

\DeclareMathOperator*{\argmax}{argmax}

\title{EECE 7205-Project 1: Max-Min-Grouping}
\author{Sreejith Sreekumar: 001277209}
\date{\today}
\begin{document}

\maketitle
\section{Problem Description}
\subsection{Input}
A one dimentional array, A.

\subsection{Problem}
Partition A into M groups, where each group is given by G[1...M] (i,e 1st group consists of G[1], 2nd group consists of G[2] etc.).
in such a way that we maximize the miniumum of the sum of these groups.

\subsection{Output}
The optimal indices at which A could be split into M groups so that we have the maximized the minimum of the group sums.


\section{Pseudo Code}

\subsection{Max-min-grouping}
\begin{spacing}{0.005}
Procedure Max-min-grouping(A, M, N):
\begin{itemize}[label={}]
\item // \textit{M is the number of groups, N is the size of A}
\item Initialize: C[M][N] $\leftarrow$ 0
\item \quad \quad \quad \quad \quad  T[M][N] $\leftarrow$ 0
\item \textit{for} j=1:M
  \begin{itemize} [label={}]
  \item \textit{for} i=j:N
    \begin{itemize} [label={}]
    \item C[j][i] = $max_{(j-1 <= k < i)}$ {min(C[k; j-1], $\sum_{m = k+1}^{i} A[m])$}
    \item T[j][i] = $\argmax_{(j-1 <= k < i)}$ {min(C[k; j-1], $\sum_{m = k+1}^{i} A[m])$}
    \end{itemize}
  \item end
  \end{itemize}
\item end
\item Traceback using T matrix to find the optimal grouping configuration of A
\item return C[M][N]
\end{itemize}

\subsection{Steps}
\begin{itemize}
\item Receive the input array  and the number of groups into which it has to be partioned.
\item Apply procedure Max-min-grouping on the array.
\item Construct the values in the matrices C[M][N]  using the formula $max_{(j-1 <= k < i)}$ {min(C[k; j-1], $\sum_{m = k+1}^{i} A[m])$}.
\item Construct the values in the matrices T[M][N]  as the values of k that maximizes the function in the previous step.
  \begin{itemize}
  \item If minimum value between C[k;j-1] and $\sum_{m = k+1}^{i} A[m])$ for the k that maximized C[j][i], is C[k;j-1], then T[j][i] = k
  \item If minimum value between C[k;j-1] and $\sum_{m = k+1}^{i} A[m])$ for the k that maximized C[j][i], is $\sum_{m = k+1}^{i} A[m])$, then T[j][i] = k-1
  \end{itemize}
\item The value C[M][N] after the procedure is the maximized value of the minimum of the group sums.
\item Trace back the values in T matrix starting from T[M][N] to find the number of elements in each partition of the output.
\item The numbers hence traced back represents the starting indices from where the input array has to be partitioned for the optimal output.
\end{itemize}

\end{spacing}


\section{Asymptotic Analysis of the Running Time}
Analyzing the program we have:\\
Construction of the C Matrix, with the formula:

\[
max_{(j-1 <= k < i)} [ min(C[k; j-1], \sum_{m = k+1}^{i} A[m]) ]
\]

we can see that the program requires two loops that iterate \textit{splits} - M times and \textit{data\_size} - N times.
for calculating the M x N elements

i.e, Worst case time complexity contributed by the outer loops = M x N \\

Now, for calculating every element, there is an inner iteration, which computes the maximum of a group of elements. This is marked by variable \textit{k}
in the code. In worst case the value of \textit{k} will be N-1. \\

In addition for calculating the \textit{min} part of the formula, there is yet another inner loop (for the second section),
marked by the variable \textit{m}, which in the worst case iterates N-1 times. \\

So, considering these steps together, we can say that the total time complexity of the fragment in the worst case is:
\[
  M x N x (N-1) x (N-1) \approx M x N x N x N  \approx M x N^{3}
\]

Hence, the worst case time complexity of the code is: M x $N^{3}$.

\section{Results}

\subsection{Run 1}

\begin{verbatim}
~/code/explore-algorithms-cpp/src/project1/src $ ./a.out 

Enter the number of elements: 12

Enter the array: 3 9 7 8 2 6 5 10 1 7 6 5
Enter number of arrays to split into: 3

C Matrix

3    12     19     27    29    35    40    50    51    58    64    69
0    3      7      12    12    16    19    23    24    29    29    34    
0    0      3      7     7     8     12    15    16    18    19    19

Parent_k Matrix

0    0    0    0    0    0    0    0    0    0    0    0    
0    1    1    2    2    2    3    3    3    5    5    5    
0    0    2    3    3    3    4    5    6    6    7    7    

Finding the number of elements to be present in each group

3	4	5	

Minimum of the sum of groups: 19

Group 1:-
3	9	7	
Group 2:-
8	2	6	5	
Group 3:-
2	6	5	10	1
\end{verbatim}


\subsection{Run 2}
\begin{verbatim}
~/code/explore-algorithms-cpp/src/project1/src $ ./a.out 

Enter the number of elements: 12

Enter the array: 3 9 7 8 2 6 5 10 1 7 6 5
Enter number of arrays to split into: 4

C Matrix

3    12    19    27    29    35    40    50    51    58    64    69    
0    3     7     12    12    16    19    23    24    29    29    34    
0    0     3     7     7     8     12    15    16    18    19    19    
0    0     0     3     3     7     7     10    11    12    14    16    

Parent_k Matrix

0    0    0    0    0    0    0    0    0    0    0    0    
0    1    1    2    2    2    3    3    3    5    5    5    
0    0    2    3    3    3    4    5    6    6    7    7    
0    0    0    3    3    4    4    6    6    7    7    9    

Finding the number of elements to be present in each group
3	3	3	3

Minimum of the sum of groups: 16

Group 1:-
3	9	7	
Group 2:-
8	2	6	
Group 3:-
8	2	6	
Group 4:-
8	2	6	
\end{verbatim}

\subsection{Run 3}
\begin{verbatim}
Enter the number of elements: 7

Enter the array: 1 2 3 4 5 6 7
Enter number of arrays to split into:4

 C Matrix 
1	3	6	10	15	21	28	
0	1	3	4	6	10	13	
0	0	1	3	4	6	7	
0	0	0	1	3	4	6	

 Parent_k Matrix 
0	0	0	0	0	0	0	
0	1	2	2	3	4	4	
0	0	2	3	4	5	5	
0	0	0	3	4	5	6	

Finding the number of elements to be present in each group
4	1	1	1

Minimum of the sum of groups: 6

Group 1:-
1	2	3	4	
Group 2:-
5	
Group 3:-
2	
Group 4:-
2	
\end{verbatim}

\section{Source Code}

\begin{minted}{C++}
#include <iostream>
#include <algorithm>
#include <vector>

/**
 * Find the index of an elment in a vector
 */
int find_index(std::vector<int> max_group, int max_element){
  int index;
  for(int i=0;i<max_group.size();i++){
    if(max_group[i] == max_element){
      index = i;
      break;
    }
				     
  }
  return index;
}


int main(){

  int splits, data_size;

  /**
   * Collect the inputs from the user
   */
  std::cout << "Enter the number of elements: ";
  std::cin >> data_size;

  int data[data_size];
  std::cout << "\n" << "Enter the array: ";

  for (int idx=0; idx < data_size; idx++)
    std::cin >> data[idx];

  std::cout << "Enter number of arrays to split into: ";
  std::cin >> splits;

  /**
   * Construction of the double array C[i][j]
   * and the matrix to which it
   */
  int C[splits][data_size];
  int parent_k[splits][data_size];

  for(int i=0;i<splits;i++){
    for(int j=0;j<data_size;j++){
      C[i][j] = 0;
      parent_k[i][j] = 0;
    }
  }

  /**
   * Constructing the first row of the matrix
   * as the sum of elements in the data array
   * until the current index (including it)
   */
  C[0][0] = data[0];
  for(int k=1; k<data_size; k++){
    C[0][k] = C[0][k-1] + data[k];
  }

  /**
   * Initialize the first column of every row  
   * in the C Matrix to 0
   */
  for(int k=1; k<splits; k++){
    C[k][0] = 0;
  }

  /**
   * Calculation for other elements
   * 
   */
  for(int j=1; j<splits; j++){

    for(int i=j; i<data_size; i++){

      /**
       * Store the min values from every k 
       */
      std::vector<int> max_group;

      std::vector<int> c_group;
      std::vector<int> sum_group;
      
      for(int k=j-1; k<i; k++ ){

	/**
	 * First element in the
	 * min part of the formula
	 */
	int element1 = C[j-1][k];

	/**
	 * Second element in the
	 * min part of the formula
	 */
	int sum_element = 0;
	for(int m=k+1; m<=i; m++)
	  sum_element += data[m];

	/**
	 * Choose the minimal element from the two elements
	 */
	c_group.push_back(element1);
	sum_group.push_back(sum_element);
	
	int min_element = element1 < sum_element ? element1 : sum_element;

	/**
	 * Push back the value into the vector
	 */
	max_group.push_back(min_element);
      }

      /**
       * Maximum element in this iteration
       */
      int max_element = *std::max_element(max_group.begin(), max_group.end());

      /**
       * Find the index of the maximum element and figure out
       * which element contributed to it.
       *
       * if it is c[k;j-1], use k which brought the largest element as value
       * of parent_k[j][i]
       *
       * if it is the sum element, use k-1 as  value for parent_k[j][i]
       */
      int _index = find_index(max_group, max_element);
      int index;
      if(c_group[_index] == max_element){
	index = _index;
      } else {
	index = _index - 1;
      }
      
      max_group.clear();
      C[j][i] = max_element;
      parent_k[j][i] = index + j;
      
    }
  }

  std::cout<<"\n C Matrix \n";
  for(int x=0;x<splits;x++){
    for(int y=0;y<data_size;y++){
      std::cout<<C[x][y]<<"\t";
    }
    std::cout<<"\n";
  }

  std::cout<<"\n Parent_k Matrix \n";
  for(int x=0;x<splits;x++){
    for(int y=0;y<data_size;y++){
      std::cout<<parent_k[x][y]<<"\t";
    }
    std::cout<<"\n";
  }

  std::cout<<"\n"<<"Finding the number of elements to be present in each group\n";

  std::vector<int> increments_in_groups;
  increments_in_groups.push_back(data_size);

  int tmp_index = data_size-1;
  for(int k=splits-1; k>0; k--){
    int next_index = parent_k[k][tmp_index];
    increments_in_groups.push_back(next_index);    
    tmp_index = next_index;
  }

  std::reverse(increments_in_groups.begin(),increments_in_groups.end());
  
  std::vector<int> groups;
  groups.push_back(increments_in_groups[0]);
  for(int i=1;i<increments_in_groups.size();i++){
    groups.push_back(increments_in_groups[i] - increments_in_groups[i-1]);
  }

  /**
   *  Number of elements in each group
   */
  for(int i=0;i<groups.size();i++){
    std::cout<<groups[i]<<"\t";
  }
  std::cout<<"\n";

  /**
   * Minimum of the sum of the groups
   */
   std::cout<<"Minimum of the sum of groups: "<<C[splits - 1][data_size - 1]<<"\n";

  /**
   * Print the groups
   */
  std::cout<<"\n";
  int start = 0;
  for(int i=0; i<groups.size(); i++){

    std::cout<<"Group "<<i+1<<":-\n";
    int end = groups[i];

    for(int j=start; j<start+end;  j++){
      std::cout<<data[j]<<"\t";
    }

    start = end;
    std::cout<<"\n";
  }   
\end{minted}  

\end{document}
