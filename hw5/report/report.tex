\documentclass{article}
%\usepackage[utf8]{inputenc}
\usepackage[a4paper, total={7in, 8in}]{geometry}
\usepackage{spverbatim}
\usepackage{minted}
\usepackage{amssymb}
\usepackage{bm}
\usepackage{graphicx}

\usepackage{amsmath}
\usepackage{halloweenmath}


\title{EECE 7205-Assignment 5}
\author{Sreejith Sreekumar: 001277209}
\date{\today}
\begin{document}

\maketitle
\section{Qn1 - Dijkstra's Algorithm}

\subsection{Code}
\begin{minted}{C++}
#include <iostream>
#include <stdio.h> 
#include <limits.h> 


void output_to_console(int distance[], int vertices) 
{ 
  int src = 0; 
  printf("Vertex \t Distance from Source\n"); 
  for (int i = 1; i < vertices; i++){
      printf("%d \t %d\n", i, distance[i]);       
    } 
}


int minimun_distance_vertex(int distance[], bool shortest_path[], int vertices) 
{ 
	
  int min = INT_MAX, min_index; 
  for (int v = 0; v < vertices; v++) {
    if (shortest_path[v] == false && distance[v] <= min) {
      min = distance[v], min_index = v;
    }
  }

  return min_index; 
} 


void dijkstra(int** graph, int src, int vertices) 
{ 
	
  int predecessor[vertices]; 
  int distance[vertices]; 
  bool shortest_path[vertices]; 

  for (int i = 0; i < vertices; i++){
    predecessor[0] = -1; 
    distance[i] = INT_MAX; 
    shortest_path[i] = false; 
  } 

  distance[src] = 0; 
  for (int count = 0; count < vertices - 1; count++){
    int u = minimun_distance_vertex(distance, shortest_path, vertices); 
    shortest_path[u] = true; 
    for (int v = 0; v < vertices; v++) {
      if (!shortest_path[v] && graph[u][v] && distance[u] + graph[u][v] < distance[v]){
	predecessor[v] = u; 
	distance[v] = distance[u] + graph[u][v]; 
      }
    }
  } 

  output_to_console(distance, vertices); 
} 


int main() 
{

  int source;
  int vertices;

  std::cout<<"Enter the number of vertices: ";
  std::cin>>vertices;

  int **_graph = new int*[vertices];

  for(int i=0; i<vertices; i++){
    _graph[i] = new int[vertices];
  }

  
  for(int i=0; i<vertices; i++){
    std::cout<<"Enter the connections for vertex "<<i<<" (Please separate weights by a space): ";
    for(int j=0; j<vertices; j++){
      std::cin>>_graph[i][j];
    }
  }

  std::cout<<"Enter the source vertex: ";
  std::cin>>source;

  dijkstra(_graph, source, vertices);
  
}
\end{minted}

\subsection{Output}

\begin{minted}{C++}
Welcome to the Emacs shell

~/code/explore-algorithms-cpp/hw5 $ g++ dijkstra.cpp 
~/code/explore-algorithms-cpp/hw5 $ ./a.out 
Enter the number of vertices: 9
Enter the connections for vertex 0 (Please separate weights by a space): 0 4 0 0 0 0 0 8 0
Enter the connections for vertex 1 (Please separate weights by a space): 4 0 8 0 0 0 0 11 0
Enter the connections for vertex 2 (Please separate weights by a space): 0 8 0 7 0 4 0 0 2
Enter the connections for vertex 3 (Please separate weights by a space): 0 0 7 0 9 14 0 0 0 
Enter the connections for vertex 4 (Please separate weights by a space): 0 0 0 9 0 10 0 0 0 
Enter the connections for vertex 5 (Please separate weights by a space): 0 0 4 0 10 0 2 0 0 
Enter the connections for vertex 6 (Please separate weights by a space): 0 0 0 14 0 2 0 1 6 
Enter the connections for vertex 7 (Please separate weights by a space): 8 11 0 0 0 0 1 0 7 
Enter the connections for vertex 8 (Please separate weights by a space): 0 0 2 0 0 0 6 7 0
Enter the source vertex: 0
Vertex 	 Distance from Source
1 	 4
2 	 12
3 	 19
4 	 21
5 	 11
6 	 9
7 	 8
8 	 14
~/code/explore-algorithms-cpp/hw5 $   
\end{minted}  

\pagebreak

\section{Qn: Bellman-Ford Algorithm}
\subsection{Code}

\begin{minted}{C++}
#include <limits.h>
#include <iostream>
 
using namespace std;
 
struct Edge{
  int source;
  int destination;
  int weight;
};


struct Graph{
  int node_count;
  int edges_count;
  struct Edge* edge;
};


void output_to_console(int distance[], int vertex_count)
{
  std::cout<<"\nNodes \tDistance from Source\n";
  for (int i = 0; i < vertex_count; ++i){
    std::cout<<i<<"\t"<<distance[i]<<"\n";
  }
}


struct Graph* build(int node_count, int edges_count)
{
  
  struct Graph* graph = (struct Graph*) malloc( sizeof(struct Graph));
  graph->node_count = node_count;
  graph->edges_count = edges_count;
 
  graph->edge = (struct Edge*) malloc(graph->edges_count * sizeof(struct Edge));
  return graph;
  
}


void bellman_ford(struct Graph* graph, int source)
{
  
  int node_count = graph->node_count;
  int shortest_path[node_count];
  int edges_count = graph->edges_count;
 
  for (int i=0; i <node_count; i++)
    shortest_path[i] = INT_MAX;
 
  shortest_path[source] = 0;
 
  for (int i=1; i<=node_count-1; i++){
    for (int j=0; j <edges_count; j++){
      int u = graph->edge[j].source;
      int v = graph->edge[j].destination;
      int weight = graph->edge[j].weight;
      if (shortest_path[u] + weight < shortest_path[v])
	shortest_path[v] = shortest_path[u] + weight;
    }
  }
 
  for (int i=0; i <edges_count; i++){
    
    int u = graph->edge[i].source;
    int v = graph->edge[i].destination;
    int weight = graph->edge[i].weight;
    
    if (shortest_path[u] + weight < shortest_path[v]){
      std::cout<<"\n Graph contains negative edge cycle\n";
    }
  }
 
  output_to_console(shortest_path, node_count);
  
}


int main()
{
  
  int vertices,edges,source;
 
  std::cout<<"Enter number of vertices:";
  std::cin>>vertices;
  
  std::cout<<"Enter number of edges:";
  std::cin>>edges;

  std::cout<<"Enter source vertex ID:";
  std::cin>>source;
 
  struct Graph* graph = build(vertices, edges); 
  std::cout<<"\nEnter edge attributes: Source, destination, weight respectively\n";
  
  for(int i=0;i<edges;i++){
    std::cout<<"Edge "<<i<<": ";
    std::cin>>graph->edge[i].source>>graph->edge[i].destination>>graph->edge[i].weight;
  }
 
  bellman_ford(graph, source);
  
}
\end{minted}

\subsection{Output}

\begin{minted}{C++}
Welcome to the Emacs shell

~/code/explore-algorithms-cpp/hw5 $ g++ bellman-ford.cpp 
~/code/explore-algorithms-cpp/hw5 $ ./a.out 
Enter number of vertices:5
Enter number of edges:10
Enter source vertex ID:0

Enter edge attributes: Source, destination, weight respectively
Edge 0: 0 1 6
Edge 1: 0 2 7
Edge 2: 1 2 8
Edge 3: 1 4 -4
Edge 4: 1 3 5
Edge 5: 3 1 -2
Edge 6: 2 3 -3
Edge 7: 2 4 9
Edge 8: 4 0 9
Edge 9: 4 3 7

Nodes 	Distance from Source
0	0
1	2
2	7
3	4
4	-2
~/code/explore-algorithms-cpp/hw5 $   
\end{minted}  

\end{document}
